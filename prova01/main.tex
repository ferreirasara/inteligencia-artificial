\documentclass[11pt, twocolumn, a4paper]{article}

\usepackage[brazil]{babel}
\usepackage[utf8]{inputenc}
\usepackage[T1]{fontenc}
\usepackage{wrapfig}
\usepackage{graphicx}
\usepackage[caption=false]{subfig}
\usepackage{enumerate}
\usepackage{multirow}
\usepackage{multicol}

\title{Resolução de problemas por meio de estratégias de busca}
\author{Sara Ferreira \\ \texttt{sara.ferreira@utp.edu.br}}
\date{09/10/2020}

\begin{document}
\maketitle

\section{Introdução}
O presente trabalho é sobre as estratégias de busca utilizadas na inteligência artificial, e alguns problemas que podem ser solucionados com essas estratégias.

Está organizado em 4 capítulos principais. O primeiro aborda dois tipos de busca, busca sem informação e com informação. Já o segundo capítulo, aborda os seguintes problemas: Cubo de Rubik, Missionários e Canibais, Problema das N Rainhas e Sudoku.

\section{Fundamentação teórica}
\subsection{Busca às cegas}
Uma estratégia de busca é dita cega se ela não leva em conta informações específicas sobre o problema a ser resolvido. Essas estratégias usam somente a informação disponível na definição do problema, geram sucessores e verificam se o estado objetivo foi atingido. Existem basicamente duas estratégias cegas para a construção e pesquisa em uma árvore de busca: Busca em Largura e Busca em Profundidade.~\cite{site-metodos-de-busca}

\subsubsection{Pesquisa em largura}
Consiste em construir uma árvore de estados a partir do estado inicial, aplicando a cada momento, todas as regras possíveis aos estados do nível mais baixo, gerando todos os estados sucessores de cada um destes estados. Assim, cada nível da árvore é completamente construído antes que qualquer nodo do próximo nível seja adicionado à árvore.~\cite{site-buscas-em-largura-e-profundidade}

A estratégia da pesquisa em largura é expandir primeiro todos os nós de menor profundidade. É uma pesquisa muito sistemática, normalmente demora muito tempo e ocupa muito espaço. A principal vantagem do algoritmo de busca em largura é que este encontra o menor caminho do nodo inicial até o nodo final mais próximo.

\subsubsection{Pesquisa em profundidade}
A estratégia da pesquisa em profundidade, ao contrário da pesquisa em largura, é expandir sempre o nó mais profundo da árvore. O algoritmo de busca em profundidade não encontra necessariamente a solução mais próxima, mas pode ser mais eficiente se o problema possui um grade número de soluções ou se a maioria dos caminhos pode levar a uma solução.

\subsubsection{Pesquisa em profundidade limitada}
É muito semelhante à busca em profundidade, mas é definido um limite de profundidade, ou seja, o algoritmo faz a pesquisa somente até certo ponto.

\subsection{Busca com informação}
Utiliza conhecimento específico além da definição do problema para encontrar soluções de forma mais eficiente do que a busca cega. Utiliza-se uma função de avaliação, para definir o quão desejável é cada nó, e então escolher o nó mais desejável, que ainda não foi expandido.~\cite{inteligencia-artificial}

\subsubsection{Busca Gulosa}
A busca gulosa expande sempre o nó que, naquele momento, parece mais próximo do objetivo. Para realizar essa medição, usa-se a heurística como função de avaliação.

Como essa estratégia leva em consideração somente o nó atual, não é uma solução ótima.

\subsubsection{A Estrela (A*)}
Faz uso de mais uma função no momento de avaliar o nó. Leva em conta o custo até o momento, mais o custo estimado do nó atual até o objetivo.

\section{Análise dos problemas}

\subsection{Cubo de Rubik}
O famoso Cubo de Rubik, ou Cubo Mágico, foi inventado por Erno Rubik em 1974, e trata-se de um cubo, com um total de 26 peças móveis. Cada face do cubo possui uma cor, e ao mover as peças, as cores ficam embaralhadas. O objetivo então, é organizar novamente as cores.

\subsubsection{Estado Inicial}
Cubo com todas as cores embaralhadas.

\subsubsection{Ações}
\begin{itemize}
    \item Mover topo sentido horário
    \item Mover topo sentido anti-horário
    \item Mover base sentido horário
    \item Mover base sentido anti-horário
    \item Mover face sentido horário
    \item Mover face sentido anti-horário
    \item Mover trás sentido horário
    \item Mover trás sentido anti-horário
    \item Mover direita sentido horário
    \item Mover direita sentido anti-horário
    \item Mover esquerda sentido horário
    \item Mover esquerda sentido anti-horário
\end{itemize}

\subsubsection{Modelo de transição}
Realizar uma ação, com o objetivo de juntar cores iguais.

\subsubsection{Custo do caminho}
Cada ação terá custo igual a 1(um).

\subsubsection{Teste do Objetivo}
Cubo com as cores organizadas.

\subsubsection{Estratégia escolhida}
Para resolver o cubo mágico, pode-se usar a estratégia A*, pois, em alguns casos, será necessário optar por um caminho pior, bagunçando algumas peças que já haviam sido organizadas, para organizar as outras, e o algoritmo A* se sai melhor nesse caso.

\subsection{Missionários e canibais}
Três canibais e três missionários estão viajando juntos e chegam à margem de um rio. Eles desejam atravessar para a outra margem para, desta forma, continuar a viagem. O único meio de transporte disponível é um barco que comporta no máximo duas pessoas. Há uma outra dificuldade: em nenhum momento o número de canibais pode ser superior ao número de missionários pois desta forma os missionários estariam em grande perigo de vida.

\subsubsection{Estado Inicial}
Três canibais e três missionários, do mesmo lado do rio.

\subsubsection{Ações}
Colocar uma ou duas pessoas no barco, e atravessar o rio.

\subsubsection{Modelo de transição}
Colocar uma ou duas pessoas no barco, sem deixar mais canibais do que missionários em uma das margens.

\subsubsection{Custo do caminho}
Cada ação terá custo igual a 1(um).

\subsubsection{Teste do Objetivo}
Três canibais e três missionários, do outro lado do rio.

\subsubsection{Estratégia escolhida}
Pode ser resolvido com a busca em largura, o que garantirá o menor caminho até a solução.

\subsection{Problema das n rainhas}
O problema das oito rainhas é o problema matemático de dispor oito rainhas em um tabuleiro de xadrez de dimensão 8x8, de forma que nenhuma delas seja atacada por outra. Para tanto, é necessário que duas damas quaisquer não estejam numa mesma linha, coluna, ou diagonal. Este é um caso específico do Problema das n rainhas, no qual temos n rainhas e um tabuleiro com n x n casas(para qualquer n >= 4).

\subsubsection{Estado Inicial}
Tabuleiro vazio.

\subsubsection{Ações}
Colocar peça na posição (x,y).

\subsubsection{Modelo de transição}
Colocar peça na posição (x,y). Essa posição deve estar vazia, e não deve estar na linha de ataque de outra peça.

\subsubsection{Custo do caminho}
Cada ação terá custo igual a 1(um).

\subsubsection{Teste do Objetivo}
Tabuleiro com n rainhas, nenhuma atacando outra.

\subsubsection{Estratégia escolhida}
Pode ser resolvido com pesquisa em profundidade, pois esse problema pode apresentar muitas soluções, e a busca em profundidade pode ser mais eficiente com problemas assim.

\subsection{Sudoku}
O Sudoku é um quebra-cabeça baseado na colocação lógica de números. O objetivo do jogo é a colocação de números de 1 a 9 em cada uma das células vazias numa grade de 9x9, constituída por 3x3 sub-grades chamadas regiões. O quebra-cabeça contém algumas pistas iniciais, que são números inseridos em algumas células, de maneira a permitir uma indução ou dedução dos números em células que estejam vazias. Cada coluna, linha e região só pode ter um número de cada um dos 1 a 9.

\subsubsection{Estado Inicial}
Grade com células vazias.

\subsubsection{Ações}
Colocar um número de 1 a 9 em uma posição (x,y) da grade.

\subsubsection{Modelo de transição}
Colocar um número de 1 a 9 em uma posição (x,y) da grade, obedecendo as seguintes regras:
\begin{itemize}
    \item O número a ser colocado não pode já ter aparecido em outra célula, na mesma linha.
    \item O número a ser colocado não pode já ter aparecido em outra célula, na mesma coluna.
    \item O número a ser colocado não pode já ter aparecido em outra célula, na mesma sub-grade.
\end{itemize}

\subsubsection{Custo do caminho}
Cada ação terá custo igual a 1(um).

\subsubsection{Teste do Objetivo}
Grade com todas as células preenchidas corretamente, com nenhum número repetido na linha, coluna ou sub-grade.

\subsubsection{Estratégia escolhida}
Pode ser utilizada a busca gulosa. Apesar de não garantir a solução ótima, o sudoku será resolvido sem maiores complicações, principalmente se for um sudoku de nível mais fácil. Para um sudoku de nível difícil, a solução poderá demorar mais para ser encontrada, visto que existirá mais "tentativas e erros".

\section{Considerações finais}
Este trabalho foi muito importante para meu conhecimento e aprofundamento dos temas abordados, além de ter-me permitido aperfeiçoar competências de investigação, seleção e organização da informação.

\bibliographystyle{acm}
\bibliography{referencias}

\end{document}